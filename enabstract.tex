Superpixel segmentation is a computer vision technology that has emerged in recent years, which can simplify image representation by clustering/segmenting image pixels into several meaningful "pixel clusters". At present, this technology has been widely used in computer vision, remote sensing technology and other application fields. Based on this background, this project intends to combine HQS (High Quality Superpixel) superpixel technology to solve the target segmentation problem on UAV high-resolution remote sensing images. The main research contents are as follows:

1. In this paper, two superpixel technologies, namely SLIC (Simple Linear Iterative Clustering) and HQS, are systematically studied, and their performance in image target segmentation is compared.

2. The HQS segmentation technology was applied to the target segmentation of UAV images and compared with SLIC. Specifically, the edge extraction method RCF (Richer Convolutional Features for edge detection) was used to extract the edge of the image to obtain the contour information required for HQS segmentation, and then the target segmentation of the image was performed by HQS, and the segmentation results were compared with the SLIC segmentation results. The results show that HQS has better edge consistency and more regular superpixel segmentation boundaries than SLIC.

3. In terms of experimental platform construction, the experimental environment in this article is based on the Ubuntu operating system and equipped with NVIDIA graphics cards. Tests and parameter adjustments are carried out on this platform, which can make full use of the computing power of the GPU to effectively test and optimize the HQS superpixel segmentation algorithm.

HQS superpixel segmentation technology has demonstrated its superior performance on public benchmark datasets, not only surpassing the existing advanced algorithms in generating superpixels with stronger regularity and higher boundary fit, but also showing competitiveness in terms of computational efficiency. In practical applications such as saliency detection, this method can more accurately highlight the salient areas in the image and retain important object boundaries, especially when using the images taken by UAVs for analysis. The paper also discusses future research directions, including further improving the regularity of superpixels in boundary areas, and exploring more elegant superpixel extraction methods to adapt to rapidly changing scenes in video sequences.
