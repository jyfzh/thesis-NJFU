超像素分割技术作为计算机视觉领域的一个新兴分支,通过将图像像素智能地组织成一些视觉上紧密相连的“像素团”,不仅简化了图像的复杂性,还为后续的图像分析任务打下了基础。这种技术由于其在图像处理和遥感领域的广泛应用,已经成为研究的热点。

本文者以无人机获取的高分辨率遥感图像为目标,引入了HQS超像素分割技术,旨在提升图像中目标物体的分割效果。研究的核心内容涉及以下几个方面:

\begin{enumerate}
\item 本文深入探讨了四种主流的超像素技术,包括SLIC、LSC、SEEDS和HQS,并对比了这些技术在图像目标分割任务中的性能表现。
\item 本文将HQS技术应用于无人机图像的目标分割,通过先使用RCF边缘提取方法来获取图像的轮廓信息,进而利用HQS技术进行目标分割。与SLIC、LSC和SEEDS等其他技术相比,HQS在边缘一致性和超像素边界的规则性方面展现出了更优的表现。
\item 为了更好地测试和优化HQS算法,研究者搭建了一个基于Ubuntu操作系统并配备NVIDIA显卡的实验平台,充分利用了GPU的计算能力。在公共基准数据集上的测试结果表明,HQS技术在超像素的生成上不仅规则性强,边界贴合度高,而且在计算效率上也具有竞争优势。特别是在显著性检测等实际应用中,HQS技术能够更准确地识别和保留图像中的显著区域和重要物体边界,这对于无人机图像分析尤为重要。
\end{enumerate}

在对HQS、SLIC、LSC和SEEDS这四种超像素分割算法进行性能评估的过程中,它们在各项指标上所展现出的不同优势。HQS算法特别在召回率和准确性上有着突出的表现,这表明它在识别图像中的可分割区域以及进行精确分割方面具有强大的能力。尽管HQS算法在识别方面表现出色,但在分类一致性上,SEEDS算法却更为出色,其较高的卡帕系数意味着SEEDS算法的分割结果与实际的分割区域有更高的匹配度。SLIC算法在召回率和平均交并比上也获得了不错的成绩,虽然其准确性略低于HQS算法,但其在分割的准确性和预测超像素与真实分割区域的重叠度上的表现依然值得认可。相较之下,LSC和SEEDS算法在本次测试中的MIoU指标上得分较低,这暗示它们在无人机图像分割任务中可能需要进一步的优化。

总的来说,HQS算法在关键性能指标上展现了其强大的性能,而SLIC算法也是一个强有力的竞争者。SEEDS算法在保持一致性方面的优势同样重要,而LSC和SEEDS算法可能需要根据特定图像类型或应用场景进行调整,以提高它们的分割效果。HQS算法在边缘一致性和超像素边界的规则性方面相比其他方法更具优势,SLIC算法在分割精度上也有良好表现,但在精确度和召回率之间找到平衡点上还有改进空间。
